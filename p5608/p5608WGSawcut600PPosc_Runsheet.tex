\documentclass[letterpaper, 10pt]{article}
\usepackage[table,xcdraw]{xcolor}
\usepackage{textcomp}
\usepackage{gensymb}
\usepackage{amsmath}  % improve math presentation
\usepackage[left=0.75in,top=0.75in,bottom=0.2in,letterpaper]{geometry}
\usepackage{setspace}
\usepackage{enumitem}


%----------------------------------------------


\begin{document}

\begin{center}
    {\Large \textbf{Biax Experiment}}\\
    {\small For current calibrations -- \texttt{gpfs/group/cjm38/default/Calibrations/}}\\
    {\footnotesize \textit{Revised: 30 Nov. 2021}}
\end{center}



\begin{table}[!ht]
	\renewcommand{\arraystretch}{1.1}
	\begin{tabular}{p{10cm} p{10cm} }
	    \textbf{Exp. Name: }p5608WGSawcut600PPosc & \textbf{Date/Time: }06/01/2022\\
	    \textbf{Operator(s): }Wood & Hydraulics start: 4896.1 \\
	    Temperature ($\degree$C):  & Hydraulics end: 4906.8 \\
	    Relative Humidity ($\%$):  & Data Logger/Control File: 16-chan \\
	\end{tabular}
\end{table} 
\vspace{-0.5cm} 

\begin{table}[!ht]
	\renewcommand{\arraystretch}{1.1}
	\begin{tabular}{p{20cm}}\textbf{Purpose/Description:} DAET oscillate PP. Effect of roughness on nonlinear elasticity of dynamically-stressed rock.  \\Sample Block Used and Thickness with \textbf{no} Sample: SDS Vessel 5x5 cm \\
	\end{tabular}
    \end{table} \vspace{-0.5cm} 

\begin{table}[!ht]
        \small
        \renewcommand{\arraystretch}{1.2}
        \begin{tabular}{ |p{7cm}| } \hline 
Material: Westerly Grainite. Sawcut. 600 grit \\Benchtop Sample Thickness (mm): 32.5 \\ \hline \end{tabular} \end{table} \vspace{-0.5cm} 

\begin{table}[ht!]
        \renewcommand{\arraystretch}{1.5}
        \begin{tabular}{ |p{2.75cm}|p{4cm}|p{3.5cm}|p{2.5cm}| p{3cm}| }
            \multicolumn{3}{l}{\textbf{\textit{Load Cells:}}} & \multicolumn{2}{l}{Contact Area: 0.0022231311 $ m^2 $}\\ \hline
            \textbf{Load cell name} & \textbf{Calibrations (mV/kN)} & \textbf{Target stress (MPa)} & \textbf{Init. Voltage} & \textbf{Volt. @ load}\\
            \hline
            44mm Solid Horiz & \begin{tabular}[c]{@{}l@{}}129.984\\ (V/MPa): 0.289\end{tabular} & 4, 9.25, 11, 13, 15, 18 & -0.996 & 0.15989, 1.67699, 2.18269, 2.76063, 3.33857, 4.20549\\ \hline44mm Solid Vert & \begin{tabular}[c]{@{}l@{}}120.364\\ (V/MPa): 0.2676\end{tabular} & 0 & 3.716 & 3.716\\ \hline
    \end{tabular}
    \end{table} \vspace{-0.5cm} 

\begin{table}[ht!]
            \renewcommand{\arraystretch}{1.5}
            \begin{tabular}{ |p{4cm}|p{5cm}|p{2.5cm}| p{4.75cm}| }
            \multicolumn{2}{l}{\textbf{\textit{Vessel Pressures:}}} & \multicolumn{2}{l}{Pore Fluid:DI H2O} \\ \hline
            \textbf{Calibrations (V/MPa)} & \textbf{Pressures (MPa)} & \textbf{Init. Voltage} & \textbf{Volt. @ load} \\ \hline\textit{\small Pc:} 0.1456 & 2, 8.25 ,10.5, 12, 13.5, 12 & -0.242 & 0.0492, 0.9592, 1.2868, 1.5052, 1.7236, 1.5052\\ \hline\textit{\small PpA:} 1.5177 & 2.6, 1.4 & -0.123 & 3.82302, 2.00178\\ \hline\textit{\small PpA:} 1.483 & 2.6 & -0.596 & 3.2598\\ \hline \end{tabular}
        \end{table} \vspace{-0.5cm} 

\begin{table}[ht!]
    \small
    \renewcommand{\arraystretch}{1.2}
    \begin{tabular}{ l l } 
        \multicolumn{2}{c}{\textbf{\textit{Displacement Transducers}}} \\
        \textbf{\textit{Name}} & \textbf{\textit{Gain (mm/V)}} \\ \hline Horiz. Load-point &  0.658 \\ \hline Vert. Load-point &  3.51 \\ \hline Horiz. On-Board &  0.416 \\ \hline  \end{tabular}
    \end{table} \vspace{-0.5cm} 

\begin{table}[!ht]
        \footnotesize
        \renewcommand{\arraystretch}{1.1}
        \begin{tabular}{ p{1cm}|p{2cm} } \rowcolor[HTML]{EFEFEF}
            \multicolumn{2}{c}{\textit{Horizontal Servo Settings} \cellcolor[HTML]{EFEFEF}} \\ \hline P: 900 & D$_{atten}$: 10 \\ \hline
        I: 80 & Feedback: 512 \\ \hline 
        D: 10 & E-gain: 800 \\ \hline 
        \multicolumn{2}{c}{\textit{Vertical Servo Settings} \cellcolor[HTML]{EFEFEF}} \\ \hline 
        P: -- & D$_{atten}$  -- \\ \hline 
        I: -- & Feedback: -- \\ \hline
        D: -- & E-gain: -- \\ \hline 
    \end{tabular} \hfill 
        \renewcommand{\arraystretch}{1.1}
        \begin{tabular}{ l|l|l } \rowcolor[HTML]{EFEFEF}
        \textit{Chilled water at HPS} & \textit{Chiller Unit} & \textit{Proc. water @ Chiller} \\ \hline 1. Temp In ($\degree$F): 56 & 6. Panel Temp ($\degree$F): 64 & 10. Temp In ($\degree$F): 78 \\ \hline 
    2. Pres. In (psi): 6 & 7. Panel Pres. (psi): 46 & 11. Pres. In (psi): 2 \\ \hline 
    3. Temp Out ($\degree$F): 76 & 8. Near Pres. In (psi): 2& 12. Temp Out ($\degree$F): 48 \\ \hline 
    4. Pres. Out (psi): 2& 9. Near Pres. Out (psi): 5& 13. Pres. Out (psi): 6 \\ \hline 
    5. Flow (lpm): 15 \\ \hline 
    \multicolumn{3}{c}{\textit{Hyd. Power Supply (HPS)} \cellcolor[HTML]{EFEFEF}} \\ \hline 
    14. Tank Temp ($\degree$C): 50.8 & 15. Temp. Out ($\degree$C): 15 & 16. Pres. Out (psi): 2700 \\ \hline 
    \end{tabular} 
    \end{table} \vspace{-0.5cm} 

\newpage 
 \textbf{Experiment Notes}
 \medskip
 {\small \begin{itemize}[label=\#]
 \setlength\itemsep{0.25em}
 	 \item 923 NS @ 4 MPa
 	 \item 1950 Pc @ 2 MPa
 	 \item 2400 flow-through. PpA @ 1 MPa
 	 \item 4400 NS to 9.25 MPa, Pc to 8.25 MPa
 	 \item 5500 PpA \& PpB to 2.6 MPa
 	 \item 8200 PpA to 1.4 MPa
 	 \item 50000 100 Hz. practice PpA oscillation. 
 	 \item 58830 run1, run2. Problems recording run2 -- repeat at end of experiment.
 	 \item 2669500 NS to 11 MPa, Pc to 10.5 MPa
 	 \item 2673000 run3, run4
 	 \item 5294480 NS to 13 MPa, Pc to 12 MPa
 	 \item 5294900 run5, run6
 	 \item 7937000 NS to 15 MPa, Pc to 13.5 MPa
 	 \item 7937200 run7, run8
 	 \item 10567300 NS to 18 Mpa, Pc to 12 MPa
 	 \item 10568000 run9, run10
 	 \item 13211050 hold for data transfer
 	 \item 13213650 Pc to 13.5 MPa, NS to 15 MPa
 	 \item 13213900 run11
 	 \item 13941950 Pc to 12 MPa, NS to 13 MPa
 	 \item 13942100 run12
 	 \item 14545175 Pc to 10.5 MPa, NS to 11 MPa
 	 \item 14545400 run13
 	 \item 15137530 Pc to 8.25 MPa, NS to 9.25 MPa
 	 \item 15137830 run14, run15
 	 \item 11770840 PpA \& PpB to 0 MPa. 
 	 \item 17771000 Pc to 0 MPa. 
 	 \item 17773000 NS to 0 MPa.
 \end{itemize}} 

 \end{document}

\end{document}