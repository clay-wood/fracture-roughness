\documentclass[letterpaper, 10pt]{article}
\usepackage[table,xcdraw]{xcolor}
\usepackage{textcomp}
\usepackage{gensymb}
\usepackage{amsmath}  % improve math presentation
\usepackage[left=0.75in,top=0.75in,bottom=0.2in,letterpaper]{geometry}
\usepackage{setspace}
\usepackage{enumitem}


%----------------------------------------------


\begin{document}

\begin{center}
    {\Large \textbf{Biax Experiment}}\\
    {\small For current calibrations -- \texttt{gpfs/group/cjm38/default/Calibrations/}}\\
    {\footnotesize \textit{Revised: 30 Nov. 2021}}
\end{center}



\begin{table}[!ht]
	\renewcommand{\arraystretch}{1.1}
	\begin{tabular}{p{10cm} p{10cm} }
	    \textbf{Exp. Name: }p5607WGSawcut600NSosc & \textbf{Date/Time: }04/01/2022\\
	    \textbf{Operator(s): }Wood & Hydraulics start: 4864.4 \\
	    Temperature ($\degree$C):  & Hydraulics end:  \\
	    Relative Humidity ($\%$):  & Data Logger/Control File: 16-chan \\
	\end{tabular}
\end{table} 
\vspace{-0.5cm} 

\begin{table}[!ht]
	\renewcommand{\arraystretch}{1.1}
	\begin{tabular}{p{20cm}}\textbf{Purpose/Description:} DAET oscillate NS. Effect of roughness on nonlinear elasticity of dynamically-stressed rock.  \\Sample Block Used and Thickness with \textbf{no} Sample: SDS Vessel 5x5 cm \\
	\end{tabular}
    \end{table} \vspace{-0.5cm} 

\begin{table}[!ht]
        \small
        \renewcommand{\arraystretch}{1.2}
        \begin{tabular}{ |p{7cm}| } \hline 
Material: Westerly Grainite. Sawcut. 600 grit \\Benchtop Sample Thickness (mm): 32.5 \\ \hline \end{tabular} \end{table} \vspace{-0.5cm} 

\begin{table}[ht!]
        \renewcommand{\arraystretch}{1.5}
        \begin{tabular}{ |p{2.75cm}|p{4cm}|p{3.5cm}|p{2.5cm}| p{3cm}| }
            \multicolumn{3}{l}{\textbf{\textit{Load Cells:}}} & \multicolumn{2}{l}{Contact Area: 0.0022231311 $ m^2 $}\\ \hline
            \textbf{Load cell name} & \textbf{Calibrations (mV/kN)} & \textbf{Target stress (MPa)} & \textbf{Init. Voltage} & \textbf{Volt. @ load}\\
            \hline
            44mm Solid Horiz & \begin{tabular}[c]{@{}l@{}}129.984\\ (V/MPa): 0.289\end{tabular} & 4, 9.25, 11, 13, 15, 18 & -0.986 & 0.16989, 1.68699, 2.19269, 2.77063, 3.34857, 4.21549\\ \hline44mm Solid Vert & \begin{tabular}[c]{@{}l@{}}120.364\\ (V/MPa): 0.2676\end{tabular} & 0 & 3.704 & 3.704\\ \hline
    \end{tabular}
    \end{table} \vspace{-0.5cm} 

\begin{table}[ht!]
            \renewcommand{\arraystretch}{1.5}
            \begin{tabular}{ |p{4cm}|p{5cm}|p{2.5cm}| p{4.75cm}| }
            \multicolumn{2}{l}{\textbf{\textit{Vessel Pressures:}}} & \multicolumn{2}{l}{Pore Fluid:DI H2O} \\ \hline
            \textbf{Calibrations (V/MPa)} & \textbf{Pressures (MPa)} & \textbf{Init. Voltage} & \textbf{Volt. @ load} \\ \hline\textit{\small Pc:} 0.1456 & 2, 8.25 ,10.5, 12, 13.5, 12 & -0.2463 & 0.0449, 0.9549, 1.2825, 1.5009, 1.7193, 1.5009\\ \hline\textit{\small PpA:} 1.5177 & 2.6, 1.4 & -0.1315 & 3.81452, 1.99328\\ \hline\textit{\small PpA:} 1.483 & 2.6 & -0.595 & 3.2608\\ \hline \end{tabular}
        \end{table} \vspace{-0.5cm} 

\begin{table}[ht!]
    \small
    \renewcommand{\arraystretch}{1.2}
    \begin{tabular}{ l l } 
        \multicolumn{2}{c}{\textbf{\textit{Displacement Transducers}}} \\
        \textbf{\textit{Name}} & \textbf{\textit{Gain (mm/V)}} \\ \hline Horiz. Load-point &  0.658 \\ \hline Vert. Load-point &  3.51 \\ \hline Horiz. On-Board &  0.416 \\ \hline  \end{tabular}
    \end{table} \vspace{-0.5cm} 

\begin{table}[!ht]
        \footnotesize
        \renewcommand{\arraystretch}{1.1}
        \begin{tabular}{ p{1cm}|p{2cm} } \rowcolor[HTML]{EFEFEF}
            \multicolumn{2}{c}{\textit{Horizontal Servo Settings} \cellcolor[HTML]{EFEFEF}} \\ \hline P: 900 & D$_{atten}$: 10 \\ \hline
        I: 80 & Feedback: 512 \\ \hline 
        D: 10 & E-gain: 800 \\ \hline 
        \multicolumn{2}{c}{\textit{Vertical Servo Settings} \cellcolor[HTML]{EFEFEF}} \\ \hline 
        P: -- & D$_{atten}$  -- \\ \hline 
        I: -- & Feedback: -- \\ \hline
        D: -- & E-gain: -- \\ \hline 
    \end{tabular} \hfill 
        \renewcommand{\arraystretch}{1.1}
        \begin{tabular}{ l|l|l } \rowcolor[HTML]{EFEFEF}
        \textit{Chilled water at HPS} & \textit{Chiller Unit} & \textit{Proc. water @ Chiller} \\ \hline 1. Temp In ($\degree$F): 58 & 6. Panel Temp ($\degree$F): 66 & 10. Temp In ($\degree$F): 80 \\ \hline 
    2. Pres. In (psi): 6 & 7. Panel Pres. (psi): 46 & 11. Pres. In (psi): 2 \\ \hline 
    3. Temp Out ($\degree$F): 76 & 8. Near Pres. In (psi): 2& 12. Temp Out ($\degree$F): 48 \\ \hline 
    4. Pres. Out (psi): 2& 9. Near Pres. Out (psi): 5& 13. Pres. Out (psi): 5 \\ \hline 
    5. Flow (lpm): 15 \\ \hline 
    \multicolumn{3}{c}{\textit{Hyd. Power Supply (HPS)} \cellcolor[HTML]{EFEFEF}} \\ \hline 
    14. Tank Temp ($\degree$C): 49 & 15. Temp. Out ($\degree$C): 15 & 16. Pres. Out (psi): 2700 \\ \hline 
    \end{tabular} 
    \end{table} \vspace{-0.5cm} 

\newpage 
 \textbf{Experiment Notes}
 \medskip
 {\small \begin{itemize}[label=\#]
 \setlength\itemsep{0.25em}
 	 \item 4000 Int. DCDT Offset (We are looking for an area where the core will not be sticking) 
 	 \item 5400 Int. DCDT Offset (we once again are looking for an area where the core will not be locked) Near 6V had the best response. 
 	 \item 77000 begin saturation
 	 \item 139000 NS to 9.25 MPa, Pc to 8.25 MPa. 
 	 \item 143000 PpB to 2.6. PpA to 2.6, 1.4 MPa.
 	 \item 146000 practice NS oscillation. 0.2, 1 MPa. 
 	 \item 149700 begin flow-through, 10 Hz
 	 \item 155700 run1, run2
 	 \item 2795000 NS to 11 MPa, Pc to 10.5 MPa
 	 \item 2795500 run3, run4
 	 \item 5480500 NS to 13 MPa, Pc to 12 MPa.
 	 \item 5480830 run5, run6
 	 \item 8135900 NS to 15 MPa, Pc to 13.5 MPa.
 	 \item 8136300 run7, run8
 	 \item 10788380 NS to 18 MPa, Pc to 12 MPa.
 	 \item 10788800 run9, run10
 	 \item 13443800 Pc to 13.5 MPa, Ns to 15 MPa
 	 \item 13444000 run11
 	 \item 14069100 Pc to 12 MPa, NS to 13 MPa
 	 \item 14069300 run12
 	 \item 14684500 Pc to 10.5 MPa, NS to 11 MPa
 	 \item 14684600 run13
 	 \item 15008600 VSX computer crashed, restarted.
 	 \item 15009070 run14. same stresses as run13. restart osc protocol
 	 \item 15624100 Pc to 8.25, NS to 9.25 MPa
 	 \item 15624300 run15
 	 \item 15775000 random Horiz. lock. 
 	 \item 16219350 PpB, PpA Pc, NS to 0. end experiment
 \end{itemize}} 

 \end{document}

\end{document}