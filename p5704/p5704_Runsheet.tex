\documentclass[letterpaper, 10pt]{article}
\usepackage[table,xcdraw]{xcolor}
\usepackage{textcomp}
\usepackage{gensymb}
\usepackage{amsmath}  % improve math presentation
\usepackage[left=0.75in,top=0.75in,bottom=0.2in,letterpaper]{geometry}
\usepackage{setspace}
\usepackage{enumitem}


%----------------------------------------------


\begin{document}

\begin{center}
    {\Large \textbf{Biax Experiment}}\\
    {\small For current calibrations -- \texttt{gpfs/group/cjm38/default/Calibrations/}}\\
    {\footnotesize \textit{Revised: 30 Nov. 2021}}
\end{center}



\begin{table}[!ht]
	\renewcommand{\arraystretch}{1.1}
	\begin{tabular}{p{10cm} p{10cm} }
	    \textbf{Exp. Name: }p5704 & \textbf{Date/Time: }15/04/2022\\
	    \textbf{Operator(s): }Wood, Ke & Hydraulics start: 5212.4 \\
	    Temperature ($\degree$C):  & Hydraulics end:  \\
	    Relative Humidity ($\%$):  & Data Logger/Control File: 16-chan \\
	\end{tabular}
\end{table} 
\vspace{-0.5cm} 

\begin{table}[!ht]
	\renewcommand{\arraystretch}{1.1}
	\begin{tabular}{p{20cm}}\textbf{Purpose/Description:} measure perm of sawcut westerly granite L-block sawcut and roughened with 80/120 grit.  \\Sample Block Used and Thickness with \textbf{no} Sample: SDS Vessel 5x5 cm \\
	\end{tabular}
    \end{table} \vspace{-0.5cm} 

\begin{table}[!ht]
        \small
        \renewcommand{\arraystretch}{1.2}
        \begin{tabular}{ |p{7cm}| } \hline 
Material: Westerly Grainite. Sawcut. 120/80 grit \\Benchtop Sample Thickness (mm): 32.5 \\ \hline \end{tabular} \end{table} \vspace{-0.5cm} 

\begin{table}[ht!]
        \renewcommand{\arraystretch}{1.5}
        \begin{tabular}{ |p{2.75cm}|p{4cm}|p{3.5cm}|p{2.5cm}| p{3cm}| }
            \multicolumn{3}{l}{\textbf{\textit{Load Cells:}}} & \multicolumn{2}{l}{Contact Area: 0.0022231311 $ m^2 $}\\ \hline
            \textbf{Load cell name} & \textbf{Calibrations (mV/kN)} & \textbf{Target stress (MPa)} & \textbf{Init. Voltage} & \textbf{Volt. @ load}\\
            \hline
            44mm Solid Horiz & \begin{tabular}[c]{@{}l@{}}129.984\\ (V/MPa): 0.289\end{tabular} & 4, 9.25, 11, 13, 15, 18 & 0.0615 & 1.21739, 2.73449, 3.24019, 3.81813, 4.39607, 5.26299\\ \hline44mm Solid Vert & \begin{tabular}[c]{@{}l@{}}120.364\\ (V/MPa): 0.2676\end{tabular} & 0 & 3.764 & 3.764\\ \hline
    \end{tabular}
    \end{table} \vspace{-0.5cm} 

\begin{table}[ht!]
            \renewcommand{\arraystretch}{1.5}
            \begin{tabular}{ |p{4cm}|p{5cm}|p{2.5cm}| p{4.75cm}| }
            \multicolumn{2}{l}{\textbf{\textit{Vessel Pressures:}}} & \multicolumn{2}{l}{Pore Fluid:H2O} \\ \hline
            \textbf{Calibrations (V/MPa)} & \textbf{Pressures (MPa)} & \textbf{Init. Voltage} & \textbf{Volt. @ load} \\ \hline\textit{\small Pc:} 0.1456 & 2, 8.25, 10.5, 12, 13.5, 12 & -0.273 & 0.0182, 0.9282, 1.2558, 1.4742, 1.6926, 1.4742\\ \hline\textit{\small PpA:} 1.5177 & 2.6, 1.4 & -0.12 & 3.82602, 2.00478\\ \hline\textit{\small PpB:} 1.483 & 2.6 & -0.566 & 3.2898\\ \hline \end{tabular}
        \end{table} \vspace{-0.5cm} 

\begin{table}[ht!]
    \small
    \renewcommand{\arraystretch}{1.2}
    \begin{tabular}{ l l } 
        \multicolumn{2}{c}{\textbf{\textit{Displacement Transducers}}} \\
        \textbf{\textit{Name}} & \textbf{\textit{Gain (mm/V)}} \\ \hline Horiz. Load-point &  0.658 \\ \hline Vert. Load-point &  3.51 \\ \hline Horiz. On-Board &  0.416 \\ \hline  \end{tabular}
    \end{table} \vspace{-0.5cm} 

\begin{table}[!ht]
        \footnotesize
        \renewcommand{\arraystretch}{1.1}
        \begin{tabular}{ p{1cm}|p{2cm} } \rowcolor[HTML]{EFEFEF}
            \multicolumn{2}{c}{\textit{Horizontal Servo Settings} \cellcolor[HTML]{EFEFEF}} \\ \hline P:  & D$_{atten}$:  \\ \hline
        I:  & Feedback:  \\ \hline 
        D:  & E-gain:  \\ \hline 
        \multicolumn{2}{c}{\textit{Vertical Servo Settings} \cellcolor[HTML]{EFEFEF}} \\ \hline 
        P:  & D$_{atten}$   \\ \hline 
        I:  & Feedback:  \\ \hline
        D:  & E-gain:  \\ \hline 
    \end{tabular} \hfill 
        \renewcommand{\arraystretch}{1.1}
        \begin{tabular}{ l|l|l } \rowcolor[HTML]{EFEFEF}
        \textit{Chilled water at HPS} & \textit{Chiller Unit} & \textit{Proc. water @ Chiller} \\ \hline 1. Temp In ($\degree$F):  & 6. Panel Temp ($\degree$F):  & 10. Temp In ($\degree$F):  \\ \hline 
    2. Pres. In (psi):  & 7. Panel Pres. (psi):  & 11. Pres. In (psi):  \\ \hline 
    3. Temp Out ($\degree$F):  & 8. Near Pres. In (psi): & 12. Temp Out ($\degree$F):  \\ \hline 
    4. Pres. Out (psi): & 9. Near Pres. Out (psi): & 13. Pres. Out (psi):  \\ \hline 
    5. Flow (lpm):  \\ \hline 
    \multicolumn{3}{c}{\textit{Hyd. Power Supply (HPS)} \cellcolor[HTML]{EFEFEF}} \\ \hline 
    14. Tank Temp ($\degree$C):  & 15. Temp. Out ($\degree$C):  & 16. Pres. Out (psi):  \\ \hline 
    \end{tabular} 
    \end{table} \vspace{-0.5cm} 

\newpage 
 \textbf{Experiment Notes}
 \medskip
 {\small \begin{itemize}[label=\#]
 \setlength\itemsep{0.25em}
 	 \item 3160 apply load to 5kN
 	 \item 5940 NS to 4 MPa
 	 \item 6000 fill vessel, Pc to 2 MPa
 	 \item 8900 PpB to 1 MPa
 	 \item 9980 flow-thru. 
 	 \item 12300 NS to 9.25 MPa, Pc to 8.25 MPa
 	 \item 34000 PpA, PpB to 2.6 MPa
 	 \item 76700 PpA to 1.4 MPa, begin flow-thru
 	 \item 143000 PpA to 2.6 MPa
 	 \item 155000 NS to 11 MPa, Pc to 10.5 MPa
 	 \item 176000 PpA to 1.4 MPa, begin flow-thru
 	 \item 244500 Ppa to 2.6 MPa
 	 \item 247500 NS to 13 MPa, Pc to 12 MPa
 	 \item 264500 PpA to 1.4 MPa, begin flow-thru
 	 \item 332400 PpA to 2.6 MPa
 	 \item 374500 NS to 15 MPa, Pc to 13.5 MPa
 	 \item 385000 PpA to 1.4 MPa, begin flow-thru
 	 \item 448000 PpA to 2.6 MPa
 	 \item 455000 NS to 18 MPa, Pc to 12 MPa
 	 \item 470000 PpA to 1.4 MPa, begin flow-thru
 	 \item 540000 PpA to 2.6 MPa
 	 \item 554000 NS to 15 MPa, Pc to 13.5 MPa
 	 \item 556500 PpA to 1.4 MPa, begin flow-thru
 	 \item 622000 PpA to 2.6 MPa
 	 \item 625500 NS to 13 MPa, Pc to 12 MPa
 	 \item 640000 PpA to 1.4 MPa, begin flow-thru
 	 \item 704000 PpA to 2.6 MPa
 	 \item 709000 NS to 11 MPa, Pc to 10.5 MPa
 	 \item 727000 PpA to 1.4 MPa, begin flow-thru
 	 \item 787000 PpA to 2.6 MPa, NS to 9.25 MPa, Pc to 8.25 MPa
 	 \item 798000 PpA to 1.4 MPa, begin flow-thru
 	 \item 861500 PpA, PpB, Pc to 0 MPa, NS to 5 kN
 \end{itemize}} 

 \end{document}

\end{document}