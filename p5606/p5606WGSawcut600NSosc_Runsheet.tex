\documentclass[letterpaper, 10pt]{article}
\usepackage[table,xcdraw]{xcolor}
\usepackage{textcomp}
\usepackage{gensymb}
\usepackage{amsmath}  % improve math presentation
\usepackage[left=0.75in,top=0.75in,bottom=0.2in,letterpaper]{geometry}
\usepackage{setspace}
\usepackage{enumitem}


%----------------------------------------------


\begin{document}

\begin{center}
    {\Large \textbf{Biax Experiment}}\\
    {\small For current calibrations -- \texttt{gpfs/group/cjm38/default/Calibrations/}}\\
    {\footnotesize \textit{Revised: 30 Nov. 2021}}
\end{center}



\begin{table}[!ht]
	\renewcommand{\arraystretch}{1.1}
	\begin{tabular}{p{10cm} p{10cm} }
	    \textbf{Exp. Name: }p5606WGSawcut600NSosc & \textbf{Date/Time: }03/01/2022\\
	    \textbf{Operator(s): }Wood & Hydraulics start: 4859.3 \\
	    Temperature ($\degree$C):  & Hydraulics end:  \\
	    Relative Humidity ($\%$):  & Data Logger/Control File: 16-chan \\
	\end{tabular}
\end{table} 
\vspace{-0.5cm} 

\begin{table}[!ht]
	\renewcommand{\arraystretch}{1.1}
	\begin{tabular}{p{20cm}}\textbf{Purpose/Description:}  \\Sample Block Used and Thickness with \textbf{no} Sample: SDS Vessel 5x5 cm \\
	\end{tabular}
    \end{table} \vspace{-0.5cm} 

\begin{table}[!ht]
        \small
        \renewcommand{\arraystretch}{1.2}
        \begin{tabular}{ |p{7cm}| } \hline 
Material: Westerly Granite, Sawcut \\Benchtop Sample Thickness (mm): 32.5 \\ \hline \end{tabular} \end{table} \vspace{-0.5cm} 

\begin{table}[ht!]
        \renewcommand{\arraystretch}{1.5}
        \begin{tabular}{ |p{2.75cm}|p{4cm}|p{3.5cm}|p{2.5cm}| p{3cm}| }
            \multicolumn{3}{l}{\textbf{\textit{Load Cells:}}} & \multicolumn{2}{l}{Contact Area: 0.0022231311 $ m^2 $}\\ \hline
            \textbf{Load cell name} & \textbf{Calibrations (mV/kN)} & \textbf{Target stress (MPa)} & \textbf{Init. Voltage} & \textbf{Volt. @ load}\\
            \hline
            44mm Solid Horiz & \begin{tabular}[c]{@{}l@{}}129.984\\ (V/MPa): 0.289\end{tabular} & 4, 9.25, 11, 13, 15, 18 & -1.005 & 0.15089, 1.66799, 2.17369, 2.75163, 3.32957, 4.19649\\ \hline44mm Solid Vert & \begin{tabular}[c]{@{}l@{}}120.364\\ (V/MPa): 0.2676\end{tabular} & 0 & 3.603 & 3.603\\ \hline
    \end{tabular}
    \end{table} \vspace{-0.5cm} 

\begin{table}[ht!]
            \renewcommand{\arraystretch}{1.5}
            \begin{tabular}{ |p{4cm}|p{5cm}|p{2.5cm}| p{4.75cm}| }
            \multicolumn{2}{l}{\textbf{\textit{Vessel Pressures:}}} & \multicolumn{2}{l}{Pore Fluid:DI H2O} \\ \hline
            \textbf{Calibrations (V/MPa)} & \textbf{Pressures (MPa)} & \textbf{Init. Voltage} & \textbf{Volt. @ load} \\ \hline\textit{\small Pc:} 0.1456 & 2, 8.25, 10.5, 12, 13.5, 12 & -0.2498 & 0.0414, 0.9514, 1.279 , 1.4974, 1.7158, 1.4974\\ \hline\textit{\small PpA:} 1.5177 & 2.6, 1.4 & -0.0532 & 3.89282, 2.07158\\ \hline\textit{\small PpA:} 1.483 & 2.6 & -0.597 & 3.2588\\ \hline \end{tabular}
        \end{table} \vspace{-0.5cm} 

\begin{table}[ht!]
    \small
    \renewcommand{\arraystretch}{1.2}
    \begin{tabular}{ l l } 
        \multicolumn{2}{c}{\textbf{\textit{Displacement Transducers}}} \\
        \textbf{\textit{Name}} & \textbf{\textit{Gain (mm/V)}} \\ \hline Horiz. Load-point &  0.658 \\ \hline Vert. Load-point &  3.51 \\ \hline Horiz. On-Board &  0.416 \\ \hline  \end{tabular}
    \end{table} \vspace{-0.5cm} 

\begin{table}[!ht]
        \footnotesize
        \renewcommand{\arraystretch}{1.1}
        \begin{tabular}{ p{1cm}|p{2cm} } \rowcolor[HTML]{EFEFEF}
            \multicolumn{2}{c}{\textit{Horizontal Servo Settings} \cellcolor[HTML]{EFEFEF}} \\ \hline P: 900 & D$_{atten}$: 10 \\ \hline
        I: 80 & Feedback: 512 \\ \hline 
        D: 10 & E-gain: 800 \\ \hline 
        \multicolumn{2}{c}{\textit{Vertical Servo Settings} \cellcolor[HTML]{EFEFEF}} \\ \hline 
        P:  & D$_{atten}$   \\ \hline 
        I:  & Feedback:  \\ \hline
        D:  & E-gain:  \\ \hline 
    \end{tabular} \hfill 
        \renewcommand{\arraystretch}{1.1}
        \begin{tabular}{ l|l|l } \rowcolor[HTML]{EFEFEF}
        \textit{Chilled water at HPS} & \textit{Chiller Unit} & \textit{Proc. water @ Chiller} \\ \hline 1. Temp In ($\degree$F):  & 6. Panel Temp ($\degree$F):  & 10. Temp In ($\degree$F):  \\ \hline 
    2. Pres. In (psi):  & 7. Panel Pres. (psi):  & 11. Pres. In (psi):  \\ \hline 
    3. Temp Out ($\degree$F):  & 8. Near Pres. In (psi): & 12. Temp Out ($\degree$F):  \\ \hline 
    4. Pres. Out (psi): & 9. Near Pres. Out (psi): & 13. Pres. Out (psi):  \\ \hline 
    5. Flow (lpm):  \\ \hline 
    \multicolumn{3}{c}{\textit{Hyd. Power Supply (HPS)} \cellcolor[HTML]{EFEFEF}} \\ \hline 
    14. Tank Temp ($\degree$C):  & 15. Temp. Out ($\degree$C):  & 16. Pres. Out (psi):  \\ \hline 
    \end{tabular} 
    \end{table} \vspace{-0.5cm} 

\newpage 
 \textbf{Experiment Notes}
 \medskip
 {\small \begin{itemize}[label=\#]
 \setlength\itemsep{0.25em}
 	 \item 70 @ 5kN. check acoustics. put on front door.
 	 \item 2340 NS to 4 MPa 
 	 \item 2880 Pc to 2 MPa
 	 \item 3050 fill PpB
 	 \item 3280 saturate sample.
 	 \item 7500 10 Hz. NS to 9.25 MPa, Pc to 8.25 MPa. Int DCDT doesn't seem to track stress increase.
 	 \item 12100 PpA \& PpB to 2.6 MPa, Int DCDT still isn't registering changes. 
 	 \item 21000 Practice NS oscillation. PpA to 1.4 MPa. Begin 0.5 MPa osc. @ 1 Hz. Int DCDT does not measure oscillation, even @ 1MPa amp.
 	 \item 39630 Remove stresses and remove door. 
 	 \item 40640 adjust int dcdt
 	 \item 
 	 \item 
 \end{itemize}} 

 \end{document}

\end{document}