\documentclass[letterpaper, 10pt]{article}
\usepackage[table,xcdraw]{xcolor}
\usepackage{textcomp}
\usepackage{gensymb}
\usepackage{amsmath}  % improve math presentation
\usepackage[left=0.75in,top=0.75in,bottom=0.2in,letterpaper]{geometry}
\usepackage{setspace}
\usepackage{enumitem}


%----------------------------------------------


\begin{document}

\begin{center}
    {\Large \textbf{Biax Experiment}}\\
    {\small For current calibrations -- \texttt{gpfs/group/cjm38/default/Calibrations/}}\\
    {\footnotesize \textit{Revised: 30 Nov. 2021}}
\end{center}



\begin{table}[!ht]
	\renewcommand{\arraystretch}{1.1}
	\begin{tabular}{p{10cm} p{10cm} }
	    \textbf{Exp. Name: }p5642WGSawtootNSosc & \textbf{Date/Time: }15/02/2022\\
	    \textbf{Operator(s): }Wood, Borate, Ke & Hydraulics start: 5017.4 \\
	    Temperature ($\degree$C): 22 & Hydraulics end: 5023.9 \\
	    Relative Humidity ($\%$): 9 & Data Logger/Control File: 16-chan \\
	\end{tabular}
\end{table} 
\vspace{-0.5cm} 

\begin{table}[!ht]
	\renewcommand{\arraystretch}{1.1}
	\begin{tabular}{p{20cm}}\textbf{Purpose/Description:} DAET oscillate NS. Effect of roughness on nonlinear elasticity of dynamically-stressed rock. \\L-block of Westerly with machined roughness. 1mm wavelength, 0.5mm amp, 0.05mm ‘random’ roughness (laser). \\Sample Block Used and Thickness with \textbf{no} Sample: SDS Vessel 5x5 cm \\
	\end{tabular}
    \end{table} \vspace{-0.5cm} 

\begin{table}[!ht]
        \small
        \renewcommand{\arraystretch}{1.2}
        \begin{tabular}{ |p{7cm}| } \hline 
Material: Westerly Granite.  \\ \hline \end{tabular} \end{table} \vspace{-0.5cm} 

\begin{table}[ht!]
        \renewcommand{\arraystretch}{1.5}
        \begin{tabular}{ |p{2.75cm}|p{4cm}|p{3.5cm}|p{2.5cm}| p{3cm}| }
            \multicolumn{3}{l}{\textbf{\textit{Load Cells:}}} & \multicolumn{2}{l}{Contact Area: 0.0022231311 $ m^2 $}\\ \hline
            \textbf{Load cell name} & \textbf{Calibrations (mV/kN)} & \textbf{Target stress (MPa)} & \textbf{Init. Voltage} & \textbf{Volt. @ load}\\
            \hline
            44mm Solid Horiz & \begin{tabular}[c]{@{}l@{}}129.954\\ (V/MPa): 0.2889\end{tabular} & 6, 9.25, 11, 13, 15, 18 & -1.016 & 0.71743, 1.65637, 2.16195, 2.73976, 3.31757, 4.18429\\ \hline44mm Solid Vert & \begin{tabular}[c]{@{}l@{}}120.364\\ (V/MPa): 0.2676\end{tabular} & 0 & 0 & 0.\\ \hline
    \end{tabular}
    \end{table} \vspace{-0.5cm} 

\begin{table}[ht!]
            \renewcommand{\arraystretch}{1.5}
            \begin{tabular}{ |p{4cm}|p{5cm}|p{2.5cm}| p{4.75cm}| }
            \multicolumn{2}{l}{\textbf{\textit{Vessel Pressures:}}} & \multicolumn{2}{l}{Pore Fluid:DI H20} \\ \hline
            \textbf{Calibrations (V/MPa)} & \textbf{Pressures (MPa)} & \textbf{Init. Voltage} & \textbf{Volt. @ load} \\ \hline\textit{\small Pc:} 0.1456 & 3, 8.25, 10.5, 12, 13.5, 12 & -0.259 & 0.1778, 0.9422, 1.2698, 1.4882, 1.7066, 1.4882\\ \hline\textit{\small PpA:} 1.5177 & 2.6, 2.4, 2.0, 1.4 & -0.1209 & 3.82512, 3.52158, 2.9145 , 2.00388\\ \hline\textit{\small PpB:} 1.483 & 2.6 & -0.567 & 3.2888\\ \hline \end{tabular}
        \end{table} \vspace{-0.5cm} 

\begin{table}[ht!]
    \small
    \renewcommand{\arraystretch}{1.2}
    \begin{tabular}{ l l } 
        \multicolumn{2}{c}{\textbf{\textit{Displacement Transducers}}} \\
        \textbf{\textit{Name}} & \textbf{\textit{Gain (mm/V)}} \\ \hline Horiz. Load-point &  0.658 \\ \hline Vert. Load-point &  3.51 \\ \hline Horiz. On-Board &  0.416 \\ \hline  \end{tabular}
    \end{table} \vspace{-0.5cm} 

\begin{table}[!ht]
        \footnotesize
        \renewcommand{\arraystretch}{1.1}
        \begin{tabular}{ p{1cm}|p{2cm} } \rowcolor[HTML]{EFEFEF}
            \multicolumn{2}{c}{\textit{Horizontal Servo Settings} \cellcolor[HTML]{EFEFEF}} \\ \hline P: 900 & D$_{atten}$: 10 \\ \hline
        I: 80 & Feedback: 512 \\ \hline 
        D: 10 & E-gain: 800 \\ \hline 
        \multicolumn{2}{c}{\textit{Vertical Servo Settings} \cellcolor[HTML]{EFEFEF}} \\ \hline 
        P: -- & D$_{atten}$  -- \\ \hline 
        I: -- & Feedback: -- \\ \hline
        D: -- & E-gain: -- \\ \hline 
    \end{tabular} \hfill 
        \renewcommand{\arraystretch}{1.1}
        \begin{tabular}{ l|l|l } \rowcolor[HTML]{EFEFEF}
        \textit{Chilled water at HPS} & \textit{Chiller Unit} & \textit{Proc. water @ Chiller} \\ \hline 1. Temp In ($\degree$F): 58 & 6. Panel Temp ($\degree$F): 66 & 10. Temp In ($\degree$F): 80 \\ \hline 
    2. Pres. In (psi): 6 & 7. Panel Pres. (psi): 47 & 11. Pres. In (psi): 2 \\ \hline 
    3. Temp Out ($\degree$F): 79 & 8. Near Pres. In (psi): 2& 12. Temp Out ($\degree$F): 48 \\ \hline 
    4. Pres. Out (psi): 2& 9. Near Pres. Out (psi): 6& 13. Pres. Out (psi): 8 \\ \hline 
    5. Flow (lpm): 15 \\ \hline 
    \multicolumn{3}{c}{\textit{Hyd. Power Supply (HPS)} \cellcolor[HTML]{EFEFEF}} \\ \hline 
    14. Tank Temp ($\degree$C):   125.5 & 15. Temp. Out ($\degree$C): 15 & 16. Pres. Out (psi): 2800 \\ \hline 
    \end{tabular} 
    \end{table} \vspace{-0.5cm} 

\newpage 
 \textbf{Experiment Notes}
 \medskip
 {\small \begin{itemize}[label=\#]
 \setlength\itemsep{0.25em}
 	 \item 2285 NS to 6 MPa
 	 \item 2540 Pc to 3 MPa
 	 \item 2770 refill PpB, empty PpA
 	 \item 3050 PpB, PpA to 2.6 MPa. Difficult to prevent flow -- very permeable
 	 \item 13000 100 Hz, NS osc. Flow rate seems to vary slightly in response to osc.
 	 \item 104200 NS to 9.25 MPa, Pc to 8.25 MPa
 	 \item 104470 refill PpB, empty PpA
 	 \item 105825 10Hz, PpB, PpA to 2.6 MPa.
 	 \item 109600 1000 Hz, NS osc. set. run1
 	 \item 9337100 refill PpB, empty PpA
 	 \item 933950 NS to 11 MPa, Pc to 10.5 MPa
 	 \item 934100 10Hz, PpB, PpA to 2.6 MPa.
 	 \item 936300 1000 Hz, NS osc. set. run2
 	 \item 1768700 NS to 13 MPa, Pc to 12 MPa. 
 	 \item 1769700 1000 Hz, NS osc. set. run3
 	 \item 2597000 NS to 15 MPa, Pc to 13.5 MPa. 
 	 \item 2598900 refill PpB, empty PpA
 	 \item 2599000 10Hz, PpB, PpA to 2.6 MPa.
 	 \item 2601500 1000 Hz, NS osc. set. run4
 	 \item 3431000 NS to 18 MPa, Pc to 12 MPa.
 	 \item 3431300 10Hz, PpB, PpA to 2.6 MPa.
 	 \item 3433000 1000 Hz, NS osc. set. run5. several electrical spikes in NS. 
 	 \item 4260500 NS to 13 MPa, Pc to 12 MPa. 
 	 \item 4260800 1000 Hz, NS osc. set. run6
 	 \item 5151700 refill PpB, empty PpA
 	 \item 5152200 NS to 9.25 MPa, Pc to 8.25 MPa
 	 \item 51528000 1000 Hz, NS osc. set. run7
 	 \item 6077790 remove PpA, PpB, Pc

 \end{itemize}} 

 \end{document}

\end{document}