\documentclass[letterpaper, 10pt]{article}
\usepackage[table,xcdraw]{xcolor}
\usepackage{textcomp}
\usepackage{gensymb}
\usepackage{amsmath}  % improve math presentation
\usepackage[left=0.75in,top=0.75in,bottom=0.2in,letterpaper]{geometry}
\usepackage{setspace}
\usepackage{enumitem}


%----------------------------------------------


\begin{document}

\begin{center}
    {\Large \textbf{Biax Experiment}}\\
    {\small For current calibrations -- \texttt{gpfs/group/cjm38/default/Calibrations/}}\\
    {\footnotesize \textit{Revised: 30 Nov. 2021}}
\end{center}



\begin{table}[!ht]
	\renewcommand{\arraystretch}{1.1}
	\begin{tabular}{p{10cm} p{10cm} }
	    \textbf{Exp. Name: }p5641WGSawtoothPerm & \textbf{Date/Time: }14/02/2022\\
	    \textbf{Operator(s): }Wood, Borate, Ke & Hydraulics start: 5015.4 \\
	    Temperature ($\degree$C): 22.4 & Hydraulics end: 5017.4 \\
	    Relative Humidity ($\%$): 12 & Data Logger/Control File: 16-chan \\
	\end{tabular}
\end{table} 
\vspace{-0.5cm} 

\begin{table}[!ht]
	\renewcommand{\arraystretch}{1.1}
	\begin{tabular}{p{20cm}}\textbf{Purpose/Description:} Measure permeability of L-block of Westerly Granite with machined roughness. \\1mm wavelength, 0.5mm amp, 0.05mm `random' roughness (laser). Load up to ~10 kN, check acoustics,\\ then flow-through test.  \\Sample Block Used and Thickness with \textbf{no} Sample: SDS Vessel 5x5 cm \\
	\end{tabular}
    \end{table} \vspace{-0.5cm} 

\begin{table}[!ht]
        \small
        \renewcommand{\arraystretch}{1.2}
        \begin{tabular}{ |p{7cm}| } \hline 
Material: Westerly Granite. Sawtooth profile. \\ \hline \end{tabular} \end{table} \vspace{-0.5cm} 

\begin{table}[ht!]
        \renewcommand{\arraystretch}{1.5}
        \begin{tabular}{ |p{2.75cm}|p{4cm}|p{3.5cm}|p{2.5cm}| p{3cm}| }
            \multicolumn{3}{l}{\textbf{\textit{Load Cells:}}} & \multicolumn{2}{l}{Contact Area: 0.0022231311 $ m^2 $}\\ \hline
            \textbf{Load cell name} & \textbf{Calibrations (mV/kN)} & \textbf{Target stress (MPa)} & \textbf{Init. Voltage} & \textbf{Volt. @ load}\\
            \hline
            44mm Solid Horiz & \begin{tabular}[c]{@{}l@{}}129.954\\ (V/MPa): 0.2889\end{tabular} & 9.25 & -1.0236 & 1.64877\\ \hline44mm Solid Vert & \begin{tabular}[c]{@{}l@{}}120.364\\ (V/MPa): 0.2676\end{tabular} & 0 & 0 & 0.\\ \hline
    \end{tabular}
    \end{table} \vspace{-0.5cm} 

\begin{table}[ht!]
            \renewcommand{\arraystretch}{1.5}
            \begin{tabular}{ |p{4cm}|p{5cm}|p{2.5cm}| p{4.75cm}| }
            \multicolumn{2}{l}{\textbf{\textit{Vessel Pressures:}}} & \multicolumn{2}{l}{Pore Fluid:DI H2O} \\ \hline
            \textbf{Calibrations (V/MPa)} & \textbf{Pressures (MPa)} & \textbf{Init. Voltage} & \textbf{Volt. @ load} \\ \hline\textit{\small Pc:} 0.1456 & 8.25 & -0.2202 & 0.981\\ \hline\textit{\small PpA:} 1.5177 & 2.6, 2.4, 2, 1.4 & -0.134 & 3.81202, 3.50848, 2.9014 , 1.99078\\ \hline\textit{\small PpB:} 1.483 & 2.6 & -0.576 & 3.2798\\ \hline \end{tabular}
        \end{table} \vspace{-0.5cm} 

\begin{table}[ht!]
    \small
    \renewcommand{\arraystretch}{1.2}
    \begin{tabular}{ l l } 
        \multicolumn{2}{c}{\textbf{\textit{Displacement Transducers}}} \\
        \textbf{\textit{Name}} & \textbf{\textit{Gain (mm/V)}} \\ \hline Horiz. Load-point &  0.658 \\ \hline Vert. Load-point &  3.51 \\ \hline Horiz. On-Board &  0.416 \\ \hline  \end{tabular}
    \end{table} \vspace{-0.5cm} 

\begin{table}[!ht]
        \footnotesize
        \renewcommand{\arraystretch}{1.1}
        \begin{tabular}{ p{1cm}|p{2cm} } \rowcolor[HTML]{EFEFEF}
            \multicolumn{2}{c}{\textit{Horizontal Servo Settings} \cellcolor[HTML]{EFEFEF}} \\ \hline P: 850 & D$_{atten}$: 10 \\ \hline
        I: 80 & Feedback: 512 \\ \hline 
        D: 10 & E-gain: 800 \\ \hline 
        \multicolumn{2}{c}{\textit{Vertical Servo Settings} \cellcolor[HTML]{EFEFEF}} \\ \hline 
        P: -- & D$_{atten}$  -- \\ \hline 
        I: -- & Feedback: -- \\ \hline
        D: -- & E-gain: -- \\ \hline 
    \end{tabular} \hfill 
        \renewcommand{\arraystretch}{1.1}
        \begin{tabular}{ l|l|l } \rowcolor[HTML]{EFEFEF}
        \textit{Chilled water at HPS} & \textit{Chiller Unit} & \textit{Proc. water @ Chiller} \\ \hline 1. Temp In ($\degree$F):  & 6. Panel Temp ($\degree$F):  & 10. Temp In ($\degree$F):  \\ \hline 
    2. Pres. In (psi):  & 7. Panel Pres. (psi):  & 11. Pres. In (psi):  \\ \hline 
    3. Temp Out ($\degree$F):  & 8. Near Pres. In (psi): & 12. Temp Out ($\degree$F):  \\ \hline 
    4. Pres. Out (psi): & 9. Near Pres. Out (psi): & 13. Pres. Out (psi):  \\ \hline 
    5. Flow (lpm):  \\ \hline 
    \multicolumn{3}{c}{\textit{Hyd. Power Supply (HPS)} \cellcolor[HTML]{EFEFEF}} \\ \hline 
    14. Tank Temp ($\degree$C):  & 15. Temp. Out ($\degree$C):  & 16. Pres. Out (psi):  \\ \hline 
    \end{tabular} 
    \end{table} \vspace{-0.5cm} 

\newpage 
 \textbf{Experiment Notes}
 \medskip
 {\small \begin{itemize}[label=\#]
 \setlength\itemsep{0.25em}
 	 \item 225 go to 10 kN, hold and check acoustics
 	 \item 1610 fill PpA, PpB
 	 \item 1750 NS to 9.25 MPa
 	 \item 1950 DCDT offset
 	 \item 3250 Pc to 8.25 MPa
 	 \item 3560 PpB to 1 MPa
 	 \item 4434 PpA to 2.4 MPa -- seems very permeable
 	 \item 5974 PpA to $\approx$ 2.0 MPa
 	 \item 7600 PpA to 1.4 MPa
 	 \item 8200 PpB, PpA to 0 MPa
 	 \item 9550 Remove Pc, NS
 \end{itemize}} 

 \end{document}

\end{document}